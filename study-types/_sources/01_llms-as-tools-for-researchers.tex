\documentclass[11pt]{article}

\begin{document}

\subsection{LLMs as Tools for Software Engineering Researchers}

LLMs can be leveraged as powerful tools to assist researchers conducting empirical studies.
They can automate various tasks such as data collection, preprocessing, and analysis.
For example, LLMs can extract relevant information from large datasets, generate summaries of research papers, and even assist in writing literature reviews.
This can significantly reduce the time and effort required by researchers, allowing them to focus on more complex aspects of their studies.

\subsubsection{LLMs as Annotators}

LLMs can serve as annotators by automatically labeling artifacts with corresponding categories for data analysis.
For example, in a study analyzing code changes in version control systems, researchers may need to categorize each individual change.
For that, they may use LLMs to analyze commit messages and categorize them into predefined labels such as bug fixes, feature additions, or refactorings.
This automation can improve the efficiency of the annotation process, which is often a labor-intensive and error-prone task when done manually.

Moreover, in qualitative data analysis, manually annotating or coding text passages is also an often time-consuming manual process.
LLMs can be used to augment human annotations, provide suggestions for new codes, or even automate the entire process.

In such tasks, LLMs have the potential to improve the accuracy and efficiency of automated labeling processes~\cite{DBLP:conf/kdd/WanSJKCNSSWYABJ24}, making them valuable tools for empirical research in software engineering.
Hybrid human-LLM annotation approaches may further increase accuracy and allow for the correction of incorrectly applied labels~\cite{DBLP:conf/chi/Wang0RMM24}.

\subsubsection{LLMs as Judges}

In empirical studies, LLMs can act as judges to evaluate the quality of software artifacts such as code, documentation, and design patterns. 
For instance, LLMs can be trained to assess code readability, adherence to coding standards, and the quality of comments. 
By providing-depending on the model configuration-objective and consistent evaluations, LLMs can help mitigate certain biases and part of the variability that human judges might introduce. 
This can lead to more reliable and reproducible results in empirical studies.
When relying on the judgment of LLMs, researchers have to make sure to build a reliable process for generating ratings that considers the non-deterministic nature of LLMs and report the intricacies of that process transparently.

\subsubsection{LLMs as Subjects}

LLMs can be used as subjects in empirical studies to simulate human behavior and interactions.
For example, researchers can use LLMs to generate responses in user studies, simulate developer interactions in collaborative coding environments, or model user feedback in software usability studies.
This approach can provide valuable insights while reducing the need to recruit human participants, which can be time-consuming and costly.
Additionally, using LLMs as subjects allows for controlled experiments with consistent and repeatable conditions.
However, it is important that researchers are aware of LLMs' inherent biases~\cite{Crowell2023} and limitations~\cite{DBLP:journals/ais/HardingDLL24} when using them as study subjects.

\subsubsection{LLMs for Synthesis}

\textbf{TODO:} E.g., literature reviews

\subsubsection{References}

\bibliographystyle{plain}
\bibliography{../../literature.bib}

\end{document}

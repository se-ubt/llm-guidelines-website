\documentclass[11pt]{article}
\usepackage[parfill]{parskip} % use newlines for paragraphs (more similar to Markdown)
\newcommand{\todo}[1]{{\textbf{TODO:}\ \textit{#1}}} % command for TODOs
\usepackage{hyperref}

\begin{document}

\subsection{Report Prompts and their Development}

\subsubsection{Recommendations}
Prompts are critical in empirical software engineering studies involving LLMs. Depending on the task, prompts may include various types of content, such as source code, execution traces, error messages, natural language descriptions, or even screenshots and other multi-modal inputs. These elements significantly influence the model’s output, and understanding how exactly they were formatted and integrated is essential for transparency and reproducibility.

Researchers \textbf{MUST} report the full text of prompts used, along with any surrounding instructions, metadata, or contextual information. The exact structure of the prompt should be described, including the order and format of each element. For example, when using code snippets, researchers should specify whether they were enclosed in markdown-style code blocks (e.g., triple backticks), if line breaks and whitespace were preserved, and whether additional annotations (e.g., comments) were included. Similarly, for other artifacts such as error messages, stack traces, or non-text elements like screenshots, researchers should explain how these were presented. If rich media was involved, such as in multi-modal models, details on how these inputs were encoded or referenced in the prompt are crucial.

When dealing with extensive or complex prompts, such as those involving large codebases or multiple error logs, researchers \textbf{MUST} describe strategies they used for handling input length constraints. Approaches might include truncating, summarizing, or splitting prompts into multiple parts. Token optimization measures, such as simplifying code formatting or removing unnecessary comments, should also be documented if applied.

In terms of strategy, prompts can vary widely based on the task design. Researchers \textbf{MUST} specify whether zero-shot, one-shot, or few-shot prompting was used. For few-shot prompts, the examples provided to the model should be clearly outlined, along with the rationale for selecting them. If multiple versions of a prompt were tested, researchers should describe how these variations were evaluated and how the final design was chosen.

In cases where prompts are generated dynamically—such as through preprocessing, template structures, or retrieval-augmented generation (RAG)—the process \textbf{MUST} be thoroughly documented. This includes explaining any automated algorithms or rules that influenced prompt generation. For studies involving human participants, where users might create or modify prompts themselves, researchers \textbf{MUST} describe how these prompts were collected and analyzed. If full disclosure is not feasible due to privacy concerns, summaries and representative examples should be provided.

To ensure full reproducibility, researchers \textbf{MUST} make all prompts and prompt variations publicly available in an online appendix, replication package, or repository. If the full set of prompts is too extensive to include in the paper itself, researchers \textbf{SHOULD} still provide representative examples and describe variations in the main body of the paper.

Prompt development is often iterative, involving collaboration between human researchers and AI tools. Researchers \textbf{SHOULD} report any instances where LLMs were used to suggest prompt refinements, as well as how those suggestions were incorporated. Furthermore, prompts may need to be revised in response to failure cases where the model produced incorrect or incomplete outputs. Iterative changes based on human feedback and pilot testing results should also be included in the documentation.

Finally, pilot testing and prompt evaluation are vital for ensuring that prompts yield reliable results. If such testing was conducted, researchers \textbf{SHOULD} summarize key insights, including how different prompt variations affected output quality and which criteria were used to finalize the prompt design.

\subsubsection{Example(s)}
A debugging study may use a prompt structured like this:

\begin{quote}
\begin{verbatim}

You are a coding assistant. Below is a Python script that fails with an error. Analyze the code and suggest a fix.
Code:
```
def divide(a, b):
    return a / b

print(divide(10, 0))
```
Error message:
ZeroDivisionError: division by zero

\end{verbatim}
\end{quote}

The study should document that the code was enclosed in triple backticks, specify whether additional context (e.g., stack traces or annotations) was included, and explain how variations of the prompt were tested.

\subsubsection{Advantages}
Providing detailed documentation of prompts enhances reproducibility and comparability. It allows other researchers to replicate the study under similar conditions, refine prompts based on documented improvements, and evaluate how different types of content (e.g., source code vs. execution traces) influence LLM behavior. This transparency also enables a better understanding of how formatting, prompt length, and structure impact results across various studies.

\subsubsection{Challenges}
One challenge is the complexity of prompts that combine multiple components, such as code, error messages, and explanatory text. Formatting differences—such as whether markdown or plain text was used—can affect how LLMs interpret inputs. Additionally, prompt length constraints may require careful management, particularly for tasks involving extensive artifacts like large codebases.

For multi-modal studies, handling non-text artifacts such as screenshots introduces additional complexity. Researchers must decide how to represent such inputs, whether by textual descriptions, image encoding, or data references. Lastly, proprietary LLMs (e.g., Copilot) may obscure certain details about internal prompt processing, limiting full transparency.

Privacy and confidentiality concerns can also hinder prompt sharing, especially when sensitive data is involved. In these cases, researchers should provide anonymized examples and summaries wherever possible.

\subsubsection{Study Types}
Reporting requirements may vary depending on the study type. For tool evaluation studies, researchers \textbf{MUST} explain how prompts were generated and structured within the tool. Controlled experiments \textbf{MUST} provide exact prompts for all conditions, while observational studies \textbf{SHOULD} summarize common prompt patterns and provide representative examples if full prompts cannot be shared.

\subsubsection{References}

\bibliographystyle{plain}
\bibliography{../../literature.bib}

\end{document}

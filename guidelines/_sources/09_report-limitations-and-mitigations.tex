\documentclass[11pt]{article}
\usepackage[parfill]{parskip} % use newlines for paragraphs (more similar to Markdown)
\usepackage{hyperref}

\begin{document}

\subsection{Report Limitations and Mitigations}

\subsubsection{Context}

\textbf{TODO}

\subsubsection{Recommendations}

\textbf{TODO:} Number of repetitions, how were repetitions aggregated?, discuss limitations and mitigations

\textbf{TODO:} Discuss what makes the results of a presented study generalizable and why they are not model-dependent. Argue why the results will likely hold for a different (future) model or the next release of the LLM-based tool that was studied (e.g., ChatGPT)?

\textbf{TODO:} Discuss aspects as such increased performance (see benchmarking guidelines) vs. increased resource consumption and non-determinism.

\textbf{TODO:} Connect guideline to study types and for each type have bullet point lists with information that MUST, SHOULD, or MAY be reported (usage of those terms according to RFC 2119~\cite{rfc2119}).

\subsubsection{Example}

\textbf{TODO}

\subsubsection{Benefits}

\textbf{TODO}

\subsubsection{Challenges}

\textbf{TODO}

\subsubsection{References}

\bibliographystyle{plain}
\bibliography{../../literature.bib}

\end{document}

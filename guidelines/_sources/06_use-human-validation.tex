\documentclass[pdftex,11pt,a4paper]{article}

% language and encoding
\usepackage[utf8]{inputenc} % set document encoding to UTF-8
\usepackage[main=english, ngerman]{babel} % adjust language depending on thesis language (affects hyphenation)
\usepackage[T1]{fontenc} % set font encoding so that special characters are displayed correctly

% layout and formatting
\usepackage[paper=a4paper, inner=30mm, outer=25mm, top=30mm, bottom=25mm]{geometry}  % set page margins
\usepackage[parfill]{parskip} % use newlines for paragraphs (more similar to Markdown)
\usepackage{xspace}  % control whitespaces
\usepackage[hyphens]{url} % line breaks in URLs
\usepackage{hyperref}
\usepackage{csquotes}% environment for quotes
\newcommand*{\enq}[1]{\enquote{{\itshape#1}}} % use italics font for quotes

% math formulas
\usepackage{amsmath}
\usepackage{amssymb} 
\usepackage{eucal} % more curly versions for \mathcal{...}
\usepackage{nicefrac} % nicer fractions
\usepackage{bm} % bold math symbols

% citations
\usepackage[square,numbers,sort&compress]{natbib}

% editorial commands
\newcommand{\todo}[1]{{\textbf{TODO:}\ \textit{#1}}} % command for TODOs
\newcommand{\comment}[1]{{\textbf{Comment:}\ \textit{#1}}} % command for review comments

% RFC 2119 (https://www.rfc-editor.org/rfc/rfc2119)
% MUST: absolute requirement
\newcommand{\must}{\textbf{MUST}\xspace}
% MUST NOT: absolute prohibition
\newcommand{\mustnot}{\textbf{MUST NOT}\xspace}
% SHOULD: there may exist valid reasons in particular circumstances to ignore a  particular item, but the full implications must be understood and carefully weighed before choosing a different course
\newcommand{\should}{\textbf{SHOULD}\xspace}
% SHOULD NOT: there may exist valid reasons in particular circumstances when the particular behavior is acceptable or even useful, but the full implications should be understood and the case carefully weighed before implementing any behavior described with this label
\newcommand{\shouldnot}{\textbf{SHOULD NOT}\xspace}
% MAY: an item is truly optional
\newcommand{\may}{\textbf{MAY}\xspace}

% command to indicate where certain information should be reported
\newcommand{\paper}{PAPER\xspace}
\newcommand{\supplementarymaterial}{SUPPLEMENTARY MATERIAL\xspace}

% configure enumerate/itemize
\usepackage[inline]{enumitem}

% commands to reference sections

% scope
\newcommand{\scope}{\href{/scope/}{Motivation and Scope}\xspace}

% study types
\newcommand{\studytypes}{\href{/study-types}{Study Types}\xspace}
\newcommand{\llmsforresearcher}{\href{/study-types/\#introduction-llms-as-tools-for-software-engineering-researchers}{LLMs as Tools for Software Engineering Researchers}\xspace}
\newcommand{\annotators}{\href{/study-types\#llms-as-annotators}{LLMs as Annotators}\xspace}
\newcommand{\judges}{\href{/study-types\#llms-as-judges}{LLMs as Judges}\xspace}
\newcommand{\synthesis}{\href{/study-types\#llms-for-synthesis}{LLMs for Synthesis}\xspace}
\newcommand{\subjects}{\href{/study-types\#llms-as-participants}{LLMs as Participants}\xspace}
\newcommand{\llmsforengineers}{\href{/study-types/\#introduction-llms-as-tools-for-software-engineers}{LLMs as Tools for Software Engineers}\xspace}
\newcommand{\llmusage}{\href{/study-types\#studying-llm-usage-in-software-engineering}{Studying LLM Usage in Software Engineering}\xspace}
\newcommand{\newtools}{\href{/study-types\#llms-for-new-software-engineering-tools}{LLMs for New Software Engineering Tools}\xspace}
\newcommand{\benchmarkingtasks}{\href{/study-types\#benchmarking-llms-for-software-engineering-tasks}{Benchmarking LLMs for Software Engineering Tasks}\xspace}

%guidelines
\newcommand{\guidelines}{\href{/guidelines}{Guidelines}\xspace}
\newcommand{\usagerole}{\href{/guidelines\#declare-llm-usage-and-role}{Declare LLM Usage and Role}\xspace}
\newcommand{\modelversion}{\href{/guidelines\#report-model-version-configuration-and-customizations}{Report Model Version, Configuration, and Customizations}\xspace}
\newcommand{\toolarchitecture}{\href{/guidelines\#report-tool-architecture-beyond-models}{Report Tool Architecture beyond Models}\xspace}
\newcommand{\humanvalidation}{\href{/guidelines\#use-human-validation-for-llm-outputs}{Use Human Validation for LLM Outputs}\xspace}
\newcommand{\prompts}{\href{/guidelines\#report-prompts-their-development-and-interaction-logs}{Report Prompts, their Development, and Interaction Logs}\xspace}
\newcommand{\openllm}{\href{/guidelines\#use-an-open-llm-as-a-baseline}{Use an Open LLM as a Baseline}\xspace}
\newcommand{\benchmarksmetrics}{\href{/guidelines\#report-suitable-baselines-benchmarks-and-metrics}{Report Suitable Baselines, Benchmarks, and Metrics}\xspace}
\newcommand{\limitationsmitigations}{\href{/guidelines\#report-limitations-and-mitigations}{Report Limitations and Mitigations}\xspace}

\begin{document}

\subsection{Use Human Validation for LLM Outputs}

\subsubsection{Recommendations}

While LLMs can automate many tasks, it is important to validate their outputs with human judgment.
For natural language processing tasks, a large-scale study has shown that LLMs have significant variation in their results, which limits their reliability as a direct substitute for human raters~\cite{DBLP:journals/corr/abs-2406-18403}. 
Human validation helps ensure the accuracy and reliability of the results, as LLMs may sometimes produce incorrect or biased outputs.
Especially in studies where LLMs are used to support researchers, human validation is generally recommended to ensure validity~\cite{DBLP:conf/chi/Wang0RMM24}.
We recommend that researchers plan for validation in humans from the outset and develop their methodology with this validation in mind.
Study reference models for comparing humans with LLMs~\cite{Schneider2025ReferenceModel} can help provide a template to ease the design process.
In some cases, a hybrid approach between human and machine-generated annotations can improve annotation efficiency.
However, researchers should use systematic approaches to decide what and how human annotations can be replaced with LLM-generated ones, such as the methods proposed by Ahmed et al.~\cite{DBLP:journals/corr/abs-2408-05534}. \comment{may be personal preference, but our recommendations should take a stronger stand (and be updated when that might change)}

When evaluating the capability of LLMs to generate SE-related artifacts, may employ human validation to complement machine-generated measures.
For example, proxies for software quality, such as code complexity or the number of code smells, may be complemented by human ratings of maintainability, readability, or understandability.
In the case of more abstract variables or psychometric measurements, human validation may be the only way of measuring a specific construct.
For example, measuring human factors such as trust, cognitive load, and comprehension levels may inherently require human evaluation.

When conducting empirical measurements, researchers should clearly define the construct that they are measuring and specify the methods used for measurement. 
Further, they should use established measurement methods and instruments that are empirically validated~\cite{DBLP:journals/fcomp/HoffmanMKL23, DBLP:conf/chi/PerrigSB23}.
Measuring a construct may require aggregating input from multiple subjects. 
For example, a study may assess inter-rater agreement using measures such as Cohen's Kappa or Krippendorff’s Alpha before aggregating ratings.
In some cases, researchers may also combine multiple measures into single composite measures.
As an example, they may evaluate both the time taken and accuracy when completing a task and aggregate them into a composite measure for the participants' overall performance.
In these cases, researchers should clearly describe their method of aggregation and document their reasoning for doing so.

When employing human validation, additional confounding factors should be controlled for, such as the level of expertise or experience with LLM-based applications or their general attitude towards AI-based tools.
\comment{we might want to say something about sampling logic: exactly how many should be sampled, and how? }
Researchers should control for these factors through methods such as stratified sampling or by categorizing participants based on experience levels.
When multiple humans are annotating the same artifact, researchers should validate the objectivity of annotations through measures of inter-rater reliability.
For instance, ``A subset of 20\% of the LLM-generated annotations was reviewed and validated by experienced software engineers to ensure accuracy. Using Cohen's Kappa, an inter-rater reliability of $\kappa = 0.90$ was reached.''

For instance, ``Model-model agreement is high, for all criteria, especially for the three large models (GPT-4, Gemini, and Claude). Table I indicates that the mean Krippendorff’s α is 0.68-0.76. 
Second, we see that human-model and human-human agreements are in similar ranges, 0.24-0.40 and 0.21-0.48
for the first three categories.''~\cite{DBLP:journals/corr/abs-2408-05534}.
\comment{IRR is not applicable in the previous examples, though: we are comparing ChatGPT's code suggestions to student qualitative feedback about how useful that was. How should people report that type of agreement?}

\subsubsection{Example(s)}

As an example, Khojah et al.~\cite{DBLP:journals/pacmse/KhojahM0N24} augmented the results of their study using human measurement.
Specifically, they asked participants to provide ratings regarding their experience, trust, perceived effectiveness and efficiency, and scenarios and lessons learned in their experience with ChatGPT.

Choudhuri et al.~\cite{DBLP:conf/icse/ChoudhuriLSGS24} evaluated the perceptions of students of their experience with ChatGPT in a controlled experiment.
They added this data to extend their results from the task performance in a series of software engineering tasks.
\comment{it might be good to expand on how this helped. In this case, I am guessing the student perceptions were a way of triangulating what the metrics reported? i.e. validating that the LLM outputs were usefuL?}

Xue et al.~\cite{DBLP:conf/icse/XueCBTH24} conducted a controlled experiment in which they evaluated the impact of ChatGPT on the performance and perceptions of students in an introductory programming course.
They employed multiple measures to judge the impact of the LLM from the perspective of humans.
In their study, they recorded the students' screens, evaluated the answers they provided in tasks, and distributed a post-study survey to get direct opinions from the students.

Hymel et al.~\cite{hymel2025analysisllmsvshuman} evaluated the capability of ChatGPT-4.0 to generate requirements documents. 
Specifically, they evaluated two requirements documents based on the same business use case, one document generated with the LLM and one document created by a human expert.
The documents were then reviewed by experts and judged in terms of alignment with the original business use case, requirements quality and whether they believed it was created by a human or an LLM.
Finally, they analyzed the influence of the participants' familiarity with AI tools on the study results.

\subsubsection{Advantages}

Incorporating human judgment in the evaluation process adds a layer of quality control and increases the trustworthiness of the study's findings, especially when explicitly reporting inter-rater reliability metrics~\cite{khraisha2024canlargelanguagemodelshumans}.

Incorporating feedback from individuals from the target population strengthens external validity by grounding study findings in real-world usage scenarios and may positively impact the transfer of study results to practice.
Researchers may uncover additional opportunities to further improve the LLM or LLM-based tool based on the reported experiences.
\comment{good point here: is the "human" in the human validation section about the researcher doing the study, or the people being studied? }

\subsubsection{Challenges}

Measurement through human validation can be challenging.
Ensuring that the operationalization of a desired construct and the method of measuring it are appropriate requires a good understanding of the studied concept and construct validity in general, and a systematic design approach for the measurement instruments~\cite{DBLP:journals/tse/SjobergB23}.

Human judgment is often very subjective and may lead to large variability between different subjects due to differences in expertise, interpretation, and biases among evaluators~\cite{DBLP:journals/pacmhci/McDonaldSF19}.
Controlling for this subjectivity will require additional rigor when analyzing the study results.

Recruiting participants as human validators will always incur additional resources compared to machine-generated measures.
Researchers must weigh the cost and time investment incurred by the recruitment process against the potential benefits for the validity of their study results.

\subsubsection{Study Types}

These guidelines apply to all study types:

\begin{itemize}
    \item \href{/study-types/#llms-as-annotators}{LLMs as Annotators}
    \item \href{/study-types/#llms-as-judges}{LLMs as Judges}
    \item \href{/study-types/#llms-for-synthesis}{LLMs for Synthesis}
    \item \href{/study-types/#llms-as-subjects}{LLMs as Subjects}
    \item \href{/study-types/#studying-llm-usage-in-software-engineering}{Studying LLM Usage in Software Engineering}
    \item \href{/study-types/#llms-for-new-software-engineering-tools}{LLMs for New Software Engineering Tools}
    \item \href{/study-types/#benchmarking-llms-for-software-engineering-tasks}{Benchmarking LLMs for Software Engineering Tasks}
\end{itemize}

When conducting their studies, researchers

\must
\begin{itemize}
    \item Clearly define the construct measured through human validation.
    \item Describe how the construct is operationalized in the study, specifying the method of measurement.
    \item Employ established and widely accepted measurement methods and instruments.
\end{itemize}

\comment{the MUST points seem like general construct validity comments, and not specific enough to LLM use?}

\should
\begin{itemize}
    \item Use empirically validated measures.
    \item Complement automated or machine-generated measures with human validation where possible.
\end{itemize}

\may
\begin{itemize}
    \item Use multiple different measures (e.g., expert ratings, surveys, task performance) for human validation.
\end{itemize}

\subsubsection{References}

\bibliographystyle{plain}
\bibliography{../../literature.bib}

\end{document}

\documentclass[11pt]{article}
\usepackage[parfill]{parskip} % use newlines for paragraphs (more similar to Markdown)
\newcommand{\todo}[1]{{\textbf{TODO:}\ \textit{#1}}} % command for TODOs
\usepackage{hyperref}

\begin{document}

\subsection{Declare LLM Usage and Role}

\subsubsection{Recommendations}

When conducting any kind of empirical study involving LLMs, it is essential to clearly declare that an LLM was used.
This is, for example, required by the ACM Policy on Authorship~\cite{ACM2023}: ``The use of generative AI tools and technologies to create content is permitted but must be fully disclosed in the Work.''
We recommend reporting the exact purpose of using an LLM in a study, the tasks it was used to automate, and the expected outcomes.


\subsubsection{Example(s)}

\todo{Maybe reuse example from ACM Policy?}


\subsubsection{Advantages}

Transparency in the usage of LLMs helps in understanding the context and scope of the study, facilitating better interpretation and comparison of results.
Beyond this declaration, we recommend authors to be explicit about the LLM version they used (see \textbf{version and date} guideline) and the LLM's exact role (see \textbf{architecture guideline}).


\subsubsection{Challenges}

We do not expect any challenges for researchers following this guideline.


\subsubsection{Study Types}

This guideline MUST be followed for all study types.


\subsubsection{References}

\bibliographystyle{plain}
\bibliography{../../literature.bib}

\end{document}

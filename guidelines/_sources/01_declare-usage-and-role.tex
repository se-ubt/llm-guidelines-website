\documentclass[11pt]{article}
\usepackage[parfill]{parskip} % use newlines for paragraphs (more similar to Markdown)
\usepackage{hyperref}
\usepackage{xspace} 
\usepackage{amsmath}

% custom commands:
\newcommand{\todo}[1]{{\textbf{TODO:}\ \textit{#1}}} % command for TODOs

% RFC 2119 (https://www.rfc-editor.org/rfc/rfc2119)
% MUST: absolute requirement
\newcommand{\must}{\textbf{MUST}\xspace}
% MUST NOT: absolute prohibition
\newcommand{\mustnot}{\textbf{MUST NOT}\xspace}
% SHOULD: there may exist valid reasons in particular circumstances to ignore a  particular item, but the full implications must be understood and carefully weighed before choosing a different course
\newcommand{\should}{\textbf{SHOULD}\xspace}
% SHOULD NOT: there may exist valid reasons in particular circumstances when the particular behavior is acceptable or even useful, but the full implications should be understood and the case carefully weighed before implementing any behavior described with this label
\newcommand{\shouldnot}{\textbf{SHOULD NOT}\xspace}
% MAY: an item is truly optional
\newcommand{\may}{\textbf{MAY}\xspace}

\begin{document}

\subsection{Declare LLM Usage and Role}

\subsubsection{Recommendations}

When conducting any kind of empirical study involving LLMs, researchers \must clearly declare that an LLM was used.
This is, for example, required by the ACM Policy on Authorship: \emph{``The use of generative AI tools and technologies to create content is permitted but must be fully disclosed in the Work''}~\cite{ACM2023}.
Researchers \should report the exact purpose of using an LLM in a study, the tasks it was used to automate, and the expected outcomes.


\subsubsection{Example(s)}

The ACM Policy on Authorship suggests to to disclose the usage of Generative AI tools in the acknowledgements section of papers, for example: \emph{``ChatGPT was utilized to generate sections of this Work, including text, tables, graphs, code, data, citations''}~\cite{ACM2023}. 
Similarly, the acknowledgements section could also be used to disclose Generative AI usage for other aspects of the research, if not explicitly described in other parts of the paper.
The ACM policy further suggests: \emph{``If you are uncertain ­about the need to disclose the use of a particular tool, err on the side of caution, and include a disclosure in the acknowledgements section of the Work''}~\cite{ACM2023}.


\subsubsection{Advantages}

Transparency in the usage of LLMs helps in understanding the context and scope of the study, facilitating better interpretation and comparison of results.
Beyond this declaration, we recommend authors to be explicit about the LLM version they used (see Section \href{/guidelines/#report-model-version-and-configuration}{Report Model Version and Configuration}) and the LLM's exact role (see Section \href{/guidelines/#report-tool-architecture-and-supplemental-data}{Report Tool Architecture and Supplemental Data}).


\subsubsection{Challenges}

We do not expect any challenges for researchers following this guideline.


\subsubsection{Study Types}

This guideline \must be followed for all study types.


\subsubsection{References}

\bibliographystyle{plain}
\bibliography{../../literature.bib}

\end{document}

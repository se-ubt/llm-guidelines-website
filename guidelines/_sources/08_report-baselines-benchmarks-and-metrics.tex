\documentclass[11pt]{article}
\usepackage[parfill]{parskip} % use newlines for paragraphs (more similar to Markdown)
\usepackage{hyperref}

\begin{document}

\subsection{Report Suitable Baselines, Benchmarks, and Metrics}

\subsubsection{Context}

\textbf{TODO}

\subsubsection{Recommendations}

\textbf{TODO:} What are suitable metrics and benchmarks for evaluating LLMs? A good starting point could be this paper~\cite{10.1145/3695988}.

\begin{enumerate}
\item pass@k (\textbf{TODO:} What are common values for k? Who came up with that metric?), but also others such as CodeBLEU, etc.
\item If a tool is analyzed, the acceptance rate of generated artifacts could be interesting (how many artifacts were accepted/rejected by the user)
\item Inter-model-agreement (related to section on open LLM as baseline): Ask different LLMs or differently considered LLMs and determine their agreement 
\item ...
\end{enumerate}

\textbf{TODO:} Maybe something along the lines of using different benchmarks? Being aware of their biases (e.g., focus on a particular programming language such as Python)?

\todo{TODO: Mention aspects such as repeating the generation to handle non-determinism}

\begin{enumerate}
\item HumanEval \url{https://github.com/openai/human-eval}
\item REPOCOD \url{https://huggingface.co/datasets/lt-asset/REPOCOD}
\item CoderEval \url{https://github.com/CoderEval/CoderEval}
\item ...
\end{enumerate}

\textbf{TODO:} Connect guideline to study types and for each type have bullet point lists with information that MUST, SHOULD, or MAY be reported (usage of those terms according to RFC 2119~\cite{rfc2119}).

\textbf{TODO:} In some cases, there might be tools/methods using ``traditional'' approaches (like static analysis) that the LLM-based approach needs to be compared with. This is what is meant by ``baselines'' in the title.

\subsubsection{Example}

\textbf{TODO}

\subsubsection{Benefits}

\textbf{TODO}

\subsubsection{Challenges}

\textbf{TODO}

\subsubsection{References}

\bibliographystyle{plain}
\bibliography{../../literature.bib}

\end{document}

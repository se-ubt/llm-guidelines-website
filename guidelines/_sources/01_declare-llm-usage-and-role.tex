\documentclass[pdftex,11pt,a4paper]{article}

% language and encoding
\usepackage[utf8]{inputenc} % set document encoding to UTF-8
\usepackage[main=english, ngerman]{babel} % adjust language depending on thesis language (affects hyphenation)
\usepackage[T1]{fontenc} % set font encoding so that special characters are displayed correctly

% layout and formatting
\usepackage[paper=a4paper, inner=30mm, outer=25mm, top=30mm, bottom=25mm]{geometry}  % set page margins
\usepackage[parfill]{parskip} % use newlines for paragraphs (more similar to Markdown)
\usepackage{xspace}  % control whitespaces
\usepackage[hyphens]{url} % line breaks in URLs
\usepackage{hyperref}
\usepackage{csquotes}% environment for quotes
\newcommand*{\enq}[1]{\enquote{{\itshape#1}}} % use italics font for quotes

% math formulas
\usepackage{amsmath}
\usepackage{amssymb} 
\usepackage{eucal} % more curly versions for \mathcal{...}
\usepackage{nicefrac} % nicer fractions
\usepackage{bm} % bold math symbols

% citations
\usepackage[square,numbers,sort&compress]{natbib}

% editorial commands
\newcommand{\todo}[1]{{\textbf{TODO:}\ \textit{#1}}} % command for TODOs
\newcommand{\comment}[1]{{\textbf{Comment:}\ \textit{#1}}} % command for review comments

% RFC 2119 (https://www.rfc-editor.org/rfc/rfc2119)
% MUST: absolute requirement
\newcommand{\must}{\textbf{MUST}\xspace}
% MUST NOT: absolute prohibition
\newcommand{\mustnot}{\textbf{MUST NOT}\xspace}
% SHOULD: there may exist valid reasons in particular circumstances to ignore a  particular item, but the full implications must be understood and carefully weighed before choosing a different course
\newcommand{\should}{\textbf{SHOULD}\xspace}
% SHOULD NOT: there may exist valid reasons in particular circumstances when the particular behavior is acceptable or even useful, but the full implications should be understood and the case carefully weighed before implementing any behavior described with this label
\newcommand{\shouldnot}{\textbf{SHOULD NOT}\xspace}
% MAY: an item is truly optional
\newcommand{\may}{\textbf{MAY}\xspace}

% command to indicate where certain information should be reported
\newcommand{\paper}{PAPER\xspace}
\newcommand{\supplementarymaterial}{SUPPLEMENTARY MATERIAL\xspace}

% configure enumerate/itemize
\usepackage[inline]{enumitem}

% commands to reference sections

% scope
\newcommand{\scope}{\href{/scope/}{Motivation and Scope}\xspace}

% study types
\newcommand{\studytypes}{\href{/study-types}{Study Types}\xspace}
\newcommand{\llmsforresearcher}{\href{/study-types/\#introduction-llms-as-tools-for-software-engineering-researchers}{LLMs as Tools for Software Engineering Researchers}\xspace}
\newcommand{\annotators}{\href{/study-types\#llms-as-annotators}{LLMs as Annotators}\xspace}
\newcommand{\judges}{\href{/study-types\#llms-as-judges}{LLMs as Judges}\xspace}
\newcommand{\synthesis}{\href{/study-types\#llms-for-synthesis}{LLMs for Synthesis}\xspace}
\newcommand{\subjects}{\href{/study-types\#llms-as-participants}{LLMs as Participants}\xspace}
\newcommand{\llmsforengineers}{\href{/study-types/\#introduction-llms-as-tools-for-software-engineers}{LLMs as Tools for Software Engineers}\xspace}
\newcommand{\llmusage}{\href{/study-types\#studying-llm-usage-in-software-engineering}{Studying LLM Usage in Software Engineering}\xspace}
\newcommand{\newtools}{\href{/study-types\#llms-for-new-software-engineering-tools}{LLMs for New Software Engineering Tools}\xspace}
\newcommand{\benchmarkingtasks}{\href{/study-types\#benchmarking-llms-for-software-engineering-tasks}{Benchmarking LLMs for Software Engineering Tasks}\xspace}

%guidelines
\newcommand{\guidelines}{\href{/guidelines}{Guidelines}\xspace}
\newcommand{\usagerole}{\href{/guidelines\#declare-llm-usage-and-role}{Declare LLM Usage and Role}\xspace}
\newcommand{\modelversion}{\href{/guidelines\#report-model-version-configuration-and-customizations}{Report Model Version, Configuration, and Customizations}\xspace}
\newcommand{\toolarchitecture}{\href{/guidelines\#report-tool-architecture-beyond-models}{Report Tool Architecture beyond Models}\xspace}
\newcommand{\humanvalidation}{\href{/guidelines\#use-human-validation-for-llm-outputs}{Use Human Validation for LLM Outputs}\xspace}
\newcommand{\prompts}{\href{/guidelines\#report-prompts-their-development-and-interaction-logs}{Report Prompts, their Development, and Interaction Logs}\xspace}
\newcommand{\openllm}{\href{/guidelines\#use-an-open-llm-as-a-baseline}{Use an Open LLM as a Baseline}\xspace}
\newcommand{\benchmarksmetrics}{\href{/guidelines\#report-suitable-baselines-benchmarks-and-metrics}{Report Suitable Baselines, Benchmarks, and Metrics}\xspace}
\newcommand{\limitationsmitigations}{\href{/guidelines\#report-limitations-and-mitigations}{Report Limitations and Mitigations}\xspace}

\begin{document}

\subsection{Declare LLM Usage and Role}

\subsubsection{Recommendations}

When conducting any kind of empirical study involving LLMs, researchers \must clearly declare that an LLM was used in a suitable section of the \paper (e.g., in the introduction or research methods section).
For authoring scientific articles, such transparency is, for example, required by the ACM Policy on Authorship: \emph{``The use of generative AI tools and technologies to create content is permitted but must be fully disclosed in the Work''}~\cite{ACM2023}.
Beyond generic authorship declarations and declarations that LLMs were used as part of the research process, researchers \should report the exact purpose of using an LLM in a study, the tasks it was used to automate, and the expected outcomes in the \paper.


\subsubsection{Example(s)}

The ACM Policy on Authorship suggests to to disclose the usage of Generative AI tools in the acknowledgements section of papers, for example: \emph{``ChatGPT was utilized to generate sections of this Work, including text, tables, graphs, code, data, citations''}~\cite{ACM2023}. 
Similarly, the acknowledgements section could also be used to disclose Generative AI usage for other aspects of the research, if not explicitly described in other parts of the paper.
The ACM policy further suggests: \emph{``If you are uncertain ­about the need to disclose the use of a particular tool, err on the side of caution, and include a disclosure in the acknowledgements section of the Work''}~\cite{ACM2023}.

An example of an LLM disclosure can be found in a recent paper by Ludos et al.~\cite{DBLP:conf/re/LubosFTGMEL24}, in which they write in the methodology section:

\begin{quote}
\it
``We conducted an LLM-based evaluation of requirements utilizing the Llama 2 language model with 70 billion parameters, fine-tuned to complete chat responses...''
\end{quote}


\subsubsection{Advantages}

Transparency in the usage of LLMs helps in understanding the context and scope of the study, facilitating better interpretation and comparison of results.
Beyond this declaration, we recommend authors to be explicit about the LLM version they used (see Section \modelversion) and the LLM's exact role (see Section \toolarchitecture).


\subsubsection{Challenges}

We do not expect any challenges for researchers following this guideline.


\subsubsection{Study Types}

This guideline \must be followed for all study types.


\subsubsection{References}

\bibliographystyle{plainnat}
\bibliography{../../literature.bib}

\end{document}

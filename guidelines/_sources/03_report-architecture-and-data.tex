\documentclass[11pt]{article}
\usepackage[parfill]{parskip} % use newlines for paragraphs (more similar to Markdown)
\newcommand{\todo}[1]{{\textbf{TODO:}\ \textit{#1}}} % command for TODOs
\usepackage{hyperref}

\begin{document}

\subsection{Report Tool Architecture and Supplemental Data}

\subsubsection{Recommendations}

Oftentimes, there is a complex layer around the LLM that preprocesses data, prepares prompts, or filters user requests.
One example is ChatGPT, which can, among others, use the GPT-4o model.
GitHub Copilot uses the same model as well, and researchers can build their own tools utilizing GPT-4o directly (e.g., via the OpenAI API).
The infrastructure around the bare model can significantly contribute to the performance of a model in a certain task.
Therefore, it is crucial that researchers clearly describe what the LLM contributes to the tool or method presented in a research paper.

\todo{Architecture (e.g., usage of RAG, agent-based architecture, etc.)}

\todo{data dump of vector database if used}

\todo{finetuning? if yes, how? also: publish data used for finetuning (if not confidential)}


\subsubsection{Example(s)}

\todo{write paragraph}


\subsubsection{Advantages}

\todo{write paragraph}


\subsubsection{Challenges}

\todo{write paragraph}


\subsubsection{Study Types}

\todo{Connect guideline to study types and for each type have bullet point lists with information that MUST, SHOULD, or MAY be reported (usage of those terms according to RFC 2119~\cite{rfc2119}).}


\subsubsection{References}

\bibliographystyle{plain}
\bibliography{../../literature.bib}

\end{document}

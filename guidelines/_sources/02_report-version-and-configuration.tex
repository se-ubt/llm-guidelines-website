\documentclass[11pt]{article}
\usepackage[parfill]{parskip} % use newlines for paragraphs (more similar to Markdown)
\usepackage{hyperref}
\usepackage{xspace}
\usepackage{amsmath}

% custom commands:
\newcommand{\todo}[1]{{\textbf{TODO:}\ \textit{#1}}} % command for TODOs
\newcommand{\comment}[1]{{\textbf{Comment:}\ \textit{#1}}} % command for review comments

% RFC 2119 (https://www.rfc-editor.org/rfc/rfc2119)
% MUST: absolute requirement
\newcommand{\must}{\textbf{MUST}\xspace}
% MUST NOT: absolute prohibition
\newcommand{\mustnot}{\textbf{MUST NOT}\xspace}
% SHOULD: there may exist valid reasons in particular circumstances to ignore a  particular item, but the full implications must be understood and carefully weighed before choosing a different course
\newcommand{\should}{\textbf{SHOULD}\xspace}
% SHOULD NOT: there may exist valid reasons in particular circumstances when the particular behavior is acceptable or even useful, but the full implications should be understood and the case carefully weighed before implementing any behavior described with this label
\newcommand{\shouldnot}{\textbf{SHOULD NOT}\xspace}
% MAY: an item is truly optional
\newcommand{\may}{\textbf{MAY}\xspace}

\begin{document}

\subsection{Report Model Version and Configuration}

\subsubsection{Recommendations}

LLMs (especially when used as a service) are frequently updated, and different versions may produce varying results.
Moreover, the model configuration and parameters influence the output generation of the models.
It is crucial to document the specific version of the LLM used in the study, along with the date when the experiments were conducted, and the exact configuration being used.
Furthermore, detailed documentation of the configuration and parameters used during the study is necessary for reproducibility. 
Our recommendations is to report:
Additionally, a thorough description of the hosting environment of the LLM or LLM-based tool should be provided, especially in studies focusing on performance or any time-sensitive measurement.

\begin{itemize}
\item Model name
\item Model version
\item The maximum token length for prompts.
\item The configured temperature that controls randomness, and all other relevant parameters that affect output generation.
\item Whether historical context was considered when generating responses.
\end{itemize}


\subsubsection{Example(s)}

For an OpenAI model, researchers might report that "A  \texttt{gpt-4} model was integrated via the Azure OpenAI Service, and configured with a temperature of 0.7, top\_p set to 0.8, and a maximum token length of 512. We used version \texttt{0125-Preview}, system fingerprint \texttt{fp\_6b68a8204b}, seed value \texttt{23487}, and ran our experiment on 10th January 2025''~\cite{OpenAI25, Azure25}.
However, there are also related challenges (see Section \href{/guidelines/#challenges-1}{Challenges}).

\todo{Talk about local or self-hosted models as well.}


\subsubsection{Advantages}

By providing this information, researchers enable others to reproduce the study under the same or similar conditions.


\subsubsection{Challenges}

Different model providers and modes of operating the models allow for varying degrees of information.
For example, OpenAI provides a model version and a system fingerprint describing the backend configuration that can also influence the output.
However, the fingerprint is indeed just intended to detect changes to the model or its configuration.
As a user, one cannot go back to a certain fingerprint.
As a beta feature, OpenAI also lets users set a seed parameter to receive ``(mostly) consistent output''~\cite{OpenAI23}.
However, the seed value does not allow for full reproducibility and the fingerprint changes frequently. 

\todo{Also ``open'' models come with challenges in terms of reproducibility (\url{https://github.com/ollama/ollama/issues/5321}).}


\subsubsection{Study Types}

\todo{Connect guideline to study types and for each type have bullet point lists with information that MUST, SHOULD, or MAY be reported (usage of those terms according to RFC 2119~\cite{rfc2119}).}

\todo{I guess when studying the usage of a tool such as ChatGPT or Copilot, it might not be possible to report all of this?}


\subsubsection{References}

\bibliographystyle{plain}
\bibliography{../../literature.bib}

\end{document}

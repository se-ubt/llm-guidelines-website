\documentclass[pdftex,11pt,a4paper]{article}

% language and encoding
\usepackage[utf8]{inputenc} % set document encoding to UTF-8
\usepackage[main=english, ngerman]{babel} % adjust language depending on thesis language (affects hyphenation)
\usepackage[T1]{fontenc} % set font encoding so that special characters are displayed correctly

% layout and formatting
\usepackage[paper=a4paper, inner=30mm, outer=25mm, top=30mm, bottom=25mm]{geometry}  % set page margins
\usepackage[parfill]{parskip} % use newlines for paragraphs (more similar to Markdown)
\usepackage{xspace}  % control whitespaces
\usepackage[hyphens]{url} % line breaks in URLs
\usepackage{hyperref}
\usepackage{csquotes}% environment for quotes
\newcommand*{\enq}[1]{\enquote{{\itshape#1}}} % use italics font for quotes

% math formulas
\usepackage{amsmath}
\usepackage{amssymb} 
\usepackage{eucal} % more curly versions for \mathcal{...}
\usepackage{nicefrac} % nicer fractions
\usepackage{bm} % bold math symbols

% citations
\usepackage[square,numbers,sort&compress]{natbib}

% editorial commands
\newcommand{\todo}[1]{{\textbf{TODO:}\ \textit{#1}}} % command for TODOs
\newcommand{\comment}[1]{{\textbf{Comment:}\ \textit{#1}}} % command for review comments

% RFC 2119 (https://www.rfc-editor.org/rfc/rfc2119)
% MUST: absolute requirement
\newcommand{\must}{\textbf{MUST}\xspace}
% MUST NOT: absolute prohibition
\newcommand{\mustnot}{\textbf{MUST NOT}\xspace}
% SHOULD: there may exist valid reasons in particular circumstances to ignore a  particular item, but the full implications must be understood and carefully weighed before choosing a different course
\newcommand{\should}{\textbf{SHOULD}\xspace}
% SHOULD NOT: there may exist valid reasons in particular circumstances when the particular behavior is acceptable or even useful, but the full implications should be understood and the case carefully weighed before implementing any behavior described with this label
\newcommand{\shouldnot}{\textbf{SHOULD NOT}\xspace}
% MAY: an item is truly optional
\newcommand{\may}{\textbf{MAY}\xspace}

% command to indicate where certain information should be reported
\newcommand{\paper}{PAPER\xspace}
\newcommand{\supplementarymaterial}{SUPPLEMENTARY MATERIAL\xspace}

% configure enumerate/itemize
\usepackage[inline]{enumitem}

% commands to reference sections

% scope
\newcommand{\scope}{\href{/scope/}{Motivation and Scope}\xspace}

% study types
\newcommand{\studytypes}{\href{/study-types}{Study Types}\xspace}
\newcommand{\llmsforresearcher}{\href{/study-types/\#introduction-llms-as-tools-for-software-engineering-researchers}{LLMs as Tools for Software Engineering Researchers}\xspace}
\newcommand{\annotators}{\href{/study-types\#llms-as-annotators}{LLMs as Annotators}\xspace}
\newcommand{\judges}{\href{/study-types\#llms-as-judges}{LLMs as Judges}\xspace}
\newcommand{\synthesis}{\href{/study-types\#llms-for-synthesis}{LLMs for Synthesis}\xspace}
\newcommand{\subjects}{\href{/study-types\#llms-as-participants}{LLMs as Participants}\xspace}
\newcommand{\llmsforengineers}{\href{/study-types/\#introduction-llms-as-tools-for-software-engineers}{LLMs as Tools for Software Engineers}\xspace}
\newcommand{\llmusage}{\href{/study-types\#studying-llm-usage-in-software-engineering}{Studying LLM Usage in Software Engineering}\xspace}
\newcommand{\newtools}{\href{/study-types\#llms-for-new-software-engineering-tools}{LLMs for New Software Engineering Tools}\xspace}
\newcommand{\benchmarkingtasks}{\href{/study-types\#benchmarking-llms-for-software-engineering-tasks}{Benchmarking LLMs for Software Engineering Tasks}\xspace}

%guidelines
\newcommand{\guidelines}{\href{/guidelines}{Guidelines}\xspace}
\newcommand{\usagerole}{\href{/guidelines\#declare-llm-usage-and-role}{Declare LLM Usage and Role}\xspace}
\newcommand{\modelversion}{\href{/guidelines\#report-model-version-configuration-and-customizations}{Report Model Version, Configuration, and Customizations}\xspace}
\newcommand{\toolarchitecture}{\href{/guidelines\#report-tool-architecture-beyond-models}{Report Tool Architecture beyond Models}\xspace}
\newcommand{\humanvalidation}{\href{/guidelines\#use-human-validation-for-llm-outputs}{Use Human Validation for LLM Outputs}\xspace}
\newcommand{\prompts}{\href{/guidelines\#report-prompts-their-development-and-interaction-logs}{Report Prompts, their Development, and Interaction Logs}\xspace}
\newcommand{\openllm}{\href{/guidelines\#use-an-open-llm-as-a-baseline}{Use an Open LLM as a Baseline}\xspace}
\newcommand{\benchmarksmetrics}{\href{/guidelines\#report-suitable-baselines-benchmarks-and-metrics}{Report Suitable Baselines, Benchmarks, and Metrics}\xspace}
\newcommand{\limitationsmitigations}{\href{/guidelines\#report-limitations-and-mitigations}{Report Limitations and Mitigations}\xspace}

\begin{document}

\subsection{Report Model Version and Configuration}

\subsubsection{Recommendations}

LLMs or LLM-based tools, especially those offered as-a-service, are frequently updated; different versions may produce varying results for the same input.
Moreover, configuration parameters such as the temperature affect content generation.
Therefore, researchers \must document the specific model or tool version used in a study, along with the date when the experiments were conducted, and the exact configuration being used.
Since default values might change over time, researchers \should always report all configuration values, even if they used the defaults.
Depending on the specific study context, additional information regarding the architecture of the tool or experiment \should be reported (see Section \href{/guidelines/#report-tool-architecture-and-supplemental-data}{Report Tool Architecture and Supplemental Data}).
Our recommendation is to report:

\begin{itemize}
\item Model/tool name.
\item Model/tool version (including a checksum if available).
\item The configured temperature that controls randomness, and all other relevant parameters that affect output generation (e.g., seed values).
\item The context window (number of tokens).
\item Whether historical context was considered when generating responses.
\end{itemize}

\subsubsection{Example(s)}

For an OpenAI model, researchers might report that ``A  \texttt{gpt-4} model was integrated via the Azure OpenAI Service, and configured with a temperature of 0.7, top\_p set to 0.8, and a maximum token length of 512. We used version \texttt{0125-Preview}, system fingerprint \texttt{fp\_6b68a8204b}, seed value \texttt{23487}, and ran our experiment on 10th January 2025''~\cite{OpenAI25, Azure25}.
Similar statements can be made for self-hosted models, for which supplementary material can report specific instructions for reproducing results.
For example, for models provisioned using \href{https://ollama.com/library/}{ollama}, one can report the specific tag and checksum of the model being used, e.g., `llama3.3, tag 70b-instruct-q8\_0, checksum d5b5e1b84868`.
Given suitable hardware, running the corresponding model in its default configuration is then as easy as executing \texttt{ollama run llama3.3:70b-instruct-q8\_0} (see Section \href{/guidelines/#use-an-open-llm-as-a-baseline}{Use an Open LLM as a Baseline}).

Kang et al.~provide a similar statement in their paper on exploring LLM-based general bug reproduction~\cite{DBLP:conf/icse/KangYY23}:

\begin{quote}
\it
``We access OpenAI Codex via its closed beta API, using the code-davinci-002 model. For Codex, we set the temperature to 0.7, and the maximum number of tokens to 256.''
\end{quote}

Our guidelines additionally suggest to report a checksum and exact dates, but otherwise this example is close to our recommendations. 

\subsubsection{Advantages}

The recommended information is a prerequisite to enable reproducibility of LLM-based studies under the same or similar conditions.
Please note that this information alone is generally not sufficient.
Therefore, depending on the specific study setup, researchers \should provide additional information about architecture and data (\href{/guidelines/#report-tool-architecture-and-supplemental-data}{Report Tool Architecture and Supplemental Data}), prompts (\href{/guidelines/#report-prompts-and-their-development}{Report Prompts and their Development}), interaction logs (\href{/guidelines//#report-interaction-logs}{Report Interaction Logs}), and specific limitations an mitigations (\href{/guidelines/#report-limitations-and-mitigations}{Report Limitations and Mitigations}).

\subsubsection{Challenges}

Different model providers and modes of operating the models allow for varying degrees of information.
For example, OpenAI provides a model version and a system fingerprint describing the backend configuration that can also influence the output.
However, the fingerprint is indeed just intended to detect changes to the model or its configuration.
As a user, one cannot go back to a certain fingerprint.
As a beta feature, OpenAI also lets users set a seed parameter to receive ``(mostly) consistent output''~\cite{OpenAI23}.
However, the seed value does not allow for full reproducibility and the fingerprint changes frequently. 
While, as motived above, open models significantly simplify re-running experiments, they also come with challenges in terms of reproducibility, as generated outputs can be inconsistent despite setting the temperature to 0 and using a seed value (see \href{https://github.com/ollama/ollama/issues/5321}{GitHub issue for Llama3}).

\subsubsection{Study Types}

This guideline \must be followed for all study types for which the researcher has access to (parts of) the model's configuration.
They \must always report the configuration that is visible to them, acknowledging the reproducibility challenges of commercial tools and models offered as-a-service. 
When \href{/study-types/#studying-llm-usage-in-software-engineering}{Studying LLM Usage in Software Engineering}, for example the usage of commercial tools such as ChatGPT or GitHub Copilot, researchers \must be as specific as possible in describing their study setup.
The model name and date \must always be reported.
In those cases, reporting other aspects such as prompts (\href{/guidelines/#report-prompts-and-their-development}{Report Prompts and their Development}) and interaction logs (\href{/guidelines//#report-interaction-logs}{Report Interaction Logs}) is essential.

\subsubsection{References}

\bibliographystyle{plain}
\bibliography{../../literature.bib}

\end{document}

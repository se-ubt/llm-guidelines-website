\documentclass[pdftex,11pt,a4paper]{article}

% language and encoding
\usepackage[utf8]{inputenc} % set document encoding to UTF-8
\usepackage[main=english, ngerman]{babel} % adjust language depending on thesis language (affects hyphenation)
\usepackage[T1]{fontenc} % set font encoding so that special characters are displayed correctly

% layout and formatting
\usepackage[paper=a4paper, inner=30mm, outer=25mm, top=30mm, bottom=25mm]{geometry}  % set page margins
\usepackage[parfill]{parskip} % use newlines for paragraphs (more similar to Markdown)
\usepackage{xspace}  % control whitespaces
\usepackage[hyphens]{url} % line breaks in URLs
\usepackage{hyperref}
\usepackage{csquotes}% environment for quotes
\newcommand*{\enq}[1]{\enquote{{\itshape#1}}} % use italics font for quotes

% math formulas
\usepackage{amsmath}
\usepackage{amssymb} 
\usepackage{eucal} % more curly versions for \mathcal{...}
\usepackage{nicefrac} % nicer fractions
\usepackage{bm} % bold math symbols

% citations
\usepackage[square,numbers,sort&compress]{natbib}

% editorial commands
\newcommand{\todo}[1]{{\textbf{TODO:}\ \textit{#1}}} % command for TODOs
\newcommand{\comment}[1]{{\textbf{Comment:}\ \textit{#1}}} % command for review comments

% RFC 2119 (https://www.rfc-editor.org/rfc/rfc2119)
% MUST: absolute requirement
\newcommand{\must}{\textbf{MUST}\xspace}
% MUST NOT: absolute prohibition
\newcommand{\mustnot}{\textbf{MUST NOT}\xspace}
% SHOULD: there may exist valid reasons in particular circumstances to ignore a  particular item, but the full implications must be understood and carefully weighed before choosing a different course
\newcommand{\should}{\textbf{SHOULD}\xspace}
% SHOULD NOT: there may exist valid reasons in particular circumstances when the particular behavior is acceptable or even useful, but the full implications should be understood and the case carefully weighed before implementing any behavior described with this label
\newcommand{\shouldnot}{\textbf{SHOULD NOT}\xspace}
% MAY: an item is truly optional
\newcommand{\may}{\textbf{MAY}\xspace}

% command to indicate where certain information should be reported
\newcommand{\paper}{PAPER\xspace}
\newcommand{\supplementarymaterial}{SUPPLEMENTARY MATERIAL\xspace}

% configure enumerate/itemize
\usepackage[inline]{enumitem}

% commands to reference sections

% scope
\newcommand{\scope}{\href{/scope/}{Motivation and Scope}\xspace}

% study types
\newcommand{\studytypes}{\href{/study-types}{Study Types}\xspace}
\newcommand{\llmsforresearcher}{\href{/study-types/\#introduction-llms-as-tools-for-software-engineering-researchers}{LLMs as Tools for Software Engineering Researchers}\xspace}
\newcommand{\annotators}{\href{/study-types\#llms-as-annotators}{LLMs as Annotators}\xspace}
\newcommand{\judges}{\href{/study-types\#llms-as-judges}{LLMs as Judges}\xspace}
\newcommand{\synthesis}{\href{/study-types\#llms-for-synthesis}{LLMs for Synthesis}\xspace}
\newcommand{\subjects}{\href{/study-types\#llms-as-participants}{LLMs as Participants}\xspace}
\newcommand{\llmsforengineers}{\href{/study-types/\#introduction-llms-as-tools-for-software-engineers}{LLMs as Tools for Software Engineers}\xspace}
\newcommand{\llmusage}{\href{/study-types\#studying-llm-usage-in-software-engineering}{Studying LLM Usage in Software Engineering}\xspace}
\newcommand{\newtools}{\href{/study-types\#llms-for-new-software-engineering-tools}{LLMs for New Software Engineering Tools}\xspace}
\newcommand{\benchmarkingtasks}{\href{/study-types\#benchmarking-llms-for-software-engineering-tasks}{Benchmarking LLMs for Software Engineering Tasks}\xspace}

%guidelines
\newcommand{\guidelines}{\href{/guidelines}{Guidelines}\xspace}
\newcommand{\usagerole}{\href{/guidelines\#declare-llm-usage-and-role}{Declare LLM Usage and Role}\xspace}
\newcommand{\modelversion}{\href{/guidelines\#report-model-version-configuration-and-customizations}{Report Model Version, Configuration, and Customizations}\xspace}
\newcommand{\toolarchitecture}{\href{/guidelines\#report-tool-architecture-beyond-models}{Report Tool Architecture beyond Models}\xspace}
\newcommand{\humanvalidation}{\href{/guidelines\#use-human-validation-for-llm-outputs}{Use Human Validation for LLM Outputs}\xspace}
\newcommand{\prompts}{\href{/guidelines\#report-prompts-their-development-and-interaction-logs}{Report Prompts, their Development, and Interaction Logs}\xspace}
\newcommand{\openllm}{\href{/guidelines\#use-an-open-llm-as-a-baseline}{Use an Open LLM as a Baseline}\xspace}
\newcommand{\benchmarksmetrics}{\href{/guidelines\#report-suitable-baselines-benchmarks-and-metrics}{Report Suitable Baselines, Benchmarks, and Metrics}\xspace}
\newcommand{\limitationsmitigations}{\href{/guidelines\#report-limitations-and-mitigations}{Report Limitations and Mitigations}\xspace}

\begin{document}

\subsection{Report Interaction Logs}

\subsubsection{Recommendations}

% When reproducibility is important and transparency is needed, researchers \should report full interaction logs, that is, all prompts and responses generated by the LLM or LLM-based tool in the context of the presented study. Reporting this is especially important when reporting a study targeting commercial SaaS solutions based on LLMs (e.g., ChatGPT) or novel tools that integrate LLMs via cloud APIs where there is even less guarantee of reproducing the state of the LLM-powered system at a later point by a reader of the study who wants to replicate the part of the study that is downstream from the text-generation. 

Previous guidelines aim to address the reproducibility problem by encouraging the reporting of full version and parameters, but this will still not be always sufficient. Indeed, LLMs can still behave non-deterministically even if decoding strategies and parameters are fixed because non-determinism can arise from batching, input preprocessing, and floating point arithmetic on GPUs ~\cite{Chann2023}. Thus, in order to establish a fixed point from which a given study can be reproducible, a study \should report the full interaction logs with a LLM if possible. 

Reporting this is especially important when reporting a study targeting commercial SaaS solutions based on LLMs (e.g., ChatGPT) or novel tools that integrate LLMs via cloud APIs where there is even less guarantee of reproducing the state of the LLM-powered system at a later point by a reader of the study who wants to replicate it. 

To intuitively explain why this can be important, consider a study in which the researchers evaluated the correctness of bug fixing capabilities of LLM-based tools and consider that the researchers only provided the prompt without the LLM answer: 

```
You are a coding assistant. Below is a Python script that fails with an error. Analyze the code and suggest a fix.
Code:

def divide(a, b):
    return a / b

print(divide(10, 0))
Error message:
ZeroDivisionError: division by zero
'''

If the paper does not report the output and just concludes that the fix was correct, the study will be missing critical information for reproducibility. The researchers might have missed some way in which the code was incorrect, or even, the code was correct but it was not written in idiomatic Python, or could have any other number of downsides, that the readers will never be able to investigate. 


In a sense, this guideline is not different than reporting transcripts of interviews in qualitative studies: there researchers also report the full interaction between the interviewers and the participants. Intuitively, this is the same because both a human participant and the OpenAI ChatGPT might {\em provide different answers} if asked the same question at two months distance; thus the transcript of the actual conversation is important to be tracked.

% \comment{the intro does not make it super clear *why* this helps reproducibility. If the AI has changed, how does the log really help us, as the answer might be different. I would argue, like in qual research, the value is to show the researchers have engaged substnatially with the respondent, i.e. that the process was not superficial}
% \comment{I also don't really see a clear distinction with report prompts and their development. This one is arguing for transcripts of the interactions? }


\subsubsection{Example(s)}

In their paper ``Investigating ChatGPT's Potential to Assist in Requirements Elicitation Processes'' \cite{ronanki2023investigating}, Ronanki et al. report the full answers of ChatGPT and they upload them in a Zenodo record \href{https://zenodo.org/records/8124936}. 

\comment{I would suggest expanding on the details a bit, to support the point of transparency made earlier. Why is this record helpful?}


\subsubsection{Advantages}

The advantage of following this guideline is the transparency and increased reproducibility of the resulting research. 

Moreover, the guideline is easy to follow. Transcripts are easy to obtain (if we continue with the mental model of a LLM as an interviewee, this is especially evident in contrast with obtaining transcripts with human users). Even for systems where the interaction is based on voice, the interaction is first translated to text using speech-to-text methods, so it can also be easily obtained. In this sense, there is no excuse for researchers not reporting full transcripts. 

One other advantage is that, while for human participants conversations often cannot be reported due to confidentiality, LLM conversations can (e.g. as of beginning of 2025, the for-profit OpenAI company allows the sharing of chat transcripts: https://openai.com/policies/sharing-publication-policy/). 


Another advantage is that a future replication paper might be able to compare future results for the same prompts with past results for the LLM.


\subsubsection{Challenges}

Given that {\em chat transcripts} are easy to generate, a study might end up with a very large appendix. Consequently, online storage might be needed. Services such as \href{https://zenodo.org}{zenodo}, \href{https://figshare.com/}{figshare}, or other similar long term storage for research artifacts \should be used in such situations.

Not all systems allow the reporting of interaction logs with the same ease. E.g. chat bot systems are easy to report the interactions with, while auto-complete systems like GitHub Copilot, will be much harder to report. Indeed, the fact that Copilot provided a recommendation during a coding session can not be replicated unless one re-creates the whole codebase state at that given point in time -- which seems quite unrealistic. One way to report that would be sharing a screencast of the coding session. Another one is to use version control to report the state of the system when a recommendation was made.

% \comment{I think actually these tools do have logging, eg. https://docs.github.com/en/copilot/troubleshooting-github-copilot/viewing-logs-for-github-copilot-in-your-environment}
% \comment{Mircea: the logs are for debugging things that don't work }


\subsubsection{Study Types}

This guideline \should be followed for all study types. 
\comment{i agree with \should }

\subsubsection{References}
\comment{capitalization is not right in the bib entries}

\bibliographystyle{plain}
\bibliography{../../literature.bib}

\end{document}

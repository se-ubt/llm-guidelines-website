\documentclass[pdftex,11pt,a4paper]{article}

% language and encoding
\usepackage[utf8]{inputenc} % set document encoding to UTF-8
\usepackage[main=english, ngerman]{babel} % adjust language depending on thesis language (affects hyphenation)
\usepackage[T1]{fontenc} % set font encoding so that special characters are displayed correctly

% layout and formatting
\usepackage[paper=a4paper, inner=30mm, outer=25mm, top=30mm, bottom=25mm]{geometry}  % set page margins
\usepackage[parfill]{parskip} % use newlines for paragraphs (more similar to Markdown)
\usepackage{xspace}  % control whitespaces
\usepackage[hyphens]{url} % line breaks in URLs
\usepackage{hyperref}
\usepackage{csquotes}% environment for quotes
\newcommand*{\enq}[1]{\enquote{{\itshape#1}}} % use italics font for quotes

% math formulas
\usepackage{amsmath}
\usepackage{amssymb} 
\usepackage{eucal} % more curly versions for \mathcal{...}
\usepackage{nicefrac} % nicer fractions
\usepackage{bm} % bold math symbols

% citations
\usepackage[square,numbers,sort&compress]{natbib}

% editorial commands
\newcommand{\todo}[1]{{\textbf{TODO:}\ \textit{#1}}} % command for TODOs
\newcommand{\comment}[1]{{\textbf{Comment:}\ \textit{#1}}} % command for review comments

% RFC 2119 (https://www.rfc-editor.org/rfc/rfc2119)
% MUST: absolute requirement
\newcommand{\must}{\textbf{MUST}\xspace}
% MUST NOT: absolute prohibition
\newcommand{\mustnot}{\textbf{MUST NOT}\xspace}
% SHOULD: there may exist valid reasons in particular circumstances to ignore a  particular item, but the full implications must be understood and carefully weighed before choosing a different course
\newcommand{\should}{\textbf{SHOULD}\xspace}
% SHOULD NOT: there may exist valid reasons in particular circumstances when the particular behavior is acceptable or even useful, but the full implications should be understood and the case carefully weighed before implementing any behavior described with this label
\newcommand{\shouldnot}{\textbf{SHOULD NOT}\xspace}
% MAY: an item is truly optional
\newcommand{\may}{\textbf{MAY}\xspace}

% command to indicate where certain information should be reported
\newcommand{\paper}{PAPER\xspace}
\newcommand{\supplementarymaterial}{SUPPLEMENTARY MATERIAL\xspace}

% configure enumerate/itemize
\usepackage[inline]{enumitem}

% commands to reference sections

% scope
\newcommand{\scope}{\href{/scope/}{Motivation and Scope}\xspace}

% study types
\newcommand{\studytypes}{\href{/study-types}{Study Types}\xspace}
\newcommand{\llmsforresearcher}{\href{/study-types/\#introduction-llms-as-tools-for-software-engineering-researchers}{LLMs as Tools for Software Engineering Researchers}\xspace}
\newcommand{\annotators}{\href{/study-types\#llms-as-annotators}{LLMs as Annotators}\xspace}
\newcommand{\judges}{\href{/study-types\#llms-as-judges}{LLMs as Judges}\xspace}
\newcommand{\synthesis}{\href{/study-types\#llms-for-synthesis}{LLMs for Synthesis}\xspace}
\newcommand{\subjects}{\href{/study-types\#llms-as-participants}{LLMs as Participants}\xspace}
\newcommand{\llmsforengineers}{\href{/study-types/\#introduction-llms-as-tools-for-software-engineers}{LLMs as Tools for Software Engineers}\xspace}
\newcommand{\llmusage}{\href{/study-types\#studying-llm-usage-in-software-engineering}{Studying LLM Usage in Software Engineering}\xspace}
\newcommand{\newtools}{\href{/study-types\#llms-for-new-software-engineering-tools}{LLMs for New Software Engineering Tools}\xspace}
\newcommand{\benchmarkingtasks}{\href{/study-types\#benchmarking-llms-for-software-engineering-tasks}{Benchmarking LLMs for Software Engineering Tasks}\xspace}

%guidelines
\newcommand{\guidelines}{\href{/guidelines}{Guidelines}\xspace}
\newcommand{\usagerole}{\href{/guidelines\#declare-llm-usage-and-role}{Declare LLM Usage and Role}\xspace}
\newcommand{\modelversion}{\href{/guidelines\#report-model-version-configuration-and-customizations}{Report Model Version, Configuration, and Customizations}\xspace}
\newcommand{\toolarchitecture}{\href{/guidelines\#report-tool-architecture-beyond-models}{Report Tool Architecture beyond Models}\xspace}
\newcommand{\humanvalidation}{\href{/guidelines\#use-human-validation-for-llm-outputs}{Use Human Validation for LLM Outputs}\xspace}
\newcommand{\prompts}{\href{/guidelines\#report-prompts-their-development-and-interaction-logs}{Report Prompts, their Development, and Interaction Logs}\xspace}
\newcommand{\openllm}{\href{/guidelines\#use-an-open-llm-as-a-baseline}{Use an Open LLM as a Baseline}\xspace}
\newcommand{\benchmarksmetrics}{\href{/guidelines\#report-suitable-baselines-benchmarks-and-metrics}{Report Suitable Baselines, Benchmarks, and Metrics}\xspace}
\newcommand{\limitationsmitigations}{\href{/guidelines\#report-limitations-and-mitigations}{Report Limitations and Mitigations}\xspace}

\begin{document}

\subsection{Use an Open LLM as a Baseline}

\subsubsection{Recommendations}

To ensure that empirical studies using Large Language Models (LLMs) in software engineering are reproducible and comparable, 
we recommend incorporating an open LLM as a baseline for analysis. This applies whether you're using LLMs to explore something or evaluating them
on on specific software engineering tasks. 
Sometimes, including an open LLM baseline might be impossible. In such cases, researchers should at least use an open LLM on an early version of their product.
Open LLMs are available for inspection and download in places like Hugging Face. Depending on the size, and the local computing power, open LLMs can be hosted on a local computer or server employing LLM management systems such as \href{https://ollama.com/}{Ollama} and \href{https://lmstudio.ai/}{LM Studio}. Otherwise, such models can also be run on cloud-based services like \href{https://together.ai/}{Together AI}.
We recommend providing a replication package to facilitate other researchers checking your work. This should have clear, step-by-step instructions on how to get the same results you did using open LLM models. This makes the research more reliable since it allows others to confirm what you found.
For example, researchers could report: "We compared our results to Meta's Code Llama, which is available on Hugging Face." and researchers shall include a link to the replication package.

The term Open when applied to an LLM can have various meanings. \cite{widder2024open} discusses three types of openness (transparency, reusability 
and extensibility) and  what openness in AI can and cannot provide. Moreover, the Open Source Initiative (OSI) \cite{OSIAI2024} provides a definition of 
open-source AI that serves as a useful framework for evaluating the openness of AI models. In simple terms, according to OSI, open-source AI means 
you have access to everything you need to use the AI, such as understanding, modifying, sharing, retraining, and recreating. Thus, researchers must 
be clear about what "open" means in their context.

Finally, we recommend reporting and analysing inter-model agreement metrics. These metrics quantify the consistency between your model's outputs 
and the baseline, thus they support the identification of potential biases or disagreement areas.  Moreover, we recommend reporting the model 
confidence scores to analyse the model uncertainty. The analysis of inter-model agreement and model confidence provides valuable insights into 
the reliability and robustness of LLM performance, allowing a deeper understanding of their capabilities and limitations.

\subsubsection{Example(s)}
\begin{itemize}
    \item Benchmarking a Proprietary LLM: Researchers want to know how good their own LLM is at writing code. They might compare it against an open LLM such as StarCoderBase.
    \item Evaluating an LLM-Powered Tool: A team developing an AI-driven code review tool might want to assess the quality of suggestions generated by both a proprietary LLM and an open alternative. Human evaluators could then independently rate the relevance and correctness of the suggestions, providing an objective measure of the tool's effectiveness.
    \item Ensuring Reproducibility with a Replication Package: A study on bug localization that uses a closed-source LLM could support Reproducibility by including a replication package. This package might contain a script that automatically reruns the same experiments using an open-source LLM—such as Llama 3—and generates a comparative report.
\end{itemize}

\subsubsection{Advantages}
\begin{itemize}
    \item	Improved Reproducibility: Researchers can independently replicate experiments.
    \item	More Objective Comparisons: Using a standardized baseline allows for more unbiased evaluations.
    \item	Greater Transparency: Open models enable the analysis of how data is processed, which supports researchers in identifying potential biases and limitations.
    \item	Long-Term Accessibility: Unlike proprietary models, which may become unavailable, open LLMs remain available for future studies.
    \item	Lower Costs: Open-source models usually have fewer licensing restrictions, which makes them more accessible to researchers with limited funding.
\end{itemize}


\subsubsection{Challenges}
\begin{itemize}
    \item Performance Differences: Open models may not always match the latest proprietary LLMs in accuracy or efficiency, making it harder to demonstrate improvements.
    \item Computational Demands: Running large open models requires hardware resources, including high-performance GPUs and significant memory.
    \item Defining "Openness": The term open is evolving—many so-called open models provide access to weights but do not disclose training data or methodologies. We are aware that the definition of an ``open'' model is actively being discussed, and many open models are essentially only ``open weight''~\cite{Gibney2024}.
    \item We consider the \emph{Open Source AI Definition} proposed by the \emph{Open Source Initiative} (OSI)~\cite{OSIAI2024} to be a first step towards defining true open-source models.
    \item Implementation Complexity: Unlike cloud-based APIs from proprietary providers, setting up and fine-tuning open models can be technically demanding due to the possible limited documentation.
\end{itemize}

\subsubsection{Study Types}
\begin{itemize}
    \item Tool Evaluation: An open LLM baseline \must be included if technically feasible. If integration is too complex, researchers \should at least report initial benchmarking results using open models.
    \item Benchmarking Studies and Controlled Experiments: An open LLM \must be one of the models evaluated.
    \item Observational Studies: If an open LLM is impossible, the researchers \should acknowledge its absence and discuss potential impacts on their findings.
    \item Qualitative Studies: If the LLM is used for exploratory data analysis or to compare alternative interpretations of results then an LLM baseline \may be reported.
\end{itemize}

\subsubsection{References}

\bibliographystyle{plainnat}
\bibliography{../../literature.bib}

\end{document}
